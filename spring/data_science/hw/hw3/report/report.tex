\documentclass[12pt]{article}
\usepackage{amsmath}
\usepackage{graphicx}
\usepackage[margin=1in]{geometry}

% Chinese name
\usepackage{xeCJK}
\setCJKmainfont{WenQuanYi Micro Hei Mono}

% Citations
\usepackage[backend=biber,style=ieee]{biblatex}
\addbibresource{report.bib}

\begin{document}
\title{Data Science Homework 3}
\author{Andr\'es Ponce,
\and
彭思安
\and
P76107116
}
\maketitle
\section{Introduction}
Current machine learning and deep learning models have shown impressive
abilities to learn patterns from data.
From object recognition~\cite{he2016deep} to text generation~\cite{brown2020language},
many different fields have been influenced by machine learning.
These models will often take a data sample as well as its label, and
constantly adjust its weights to minimize the loss function.
Machine learning models often require large amounts of data, which they
use to learn the patterns that will be used when they are tested.
A constant issue with current models is obtaining a large enough amount
of data which accurately represents the data the model will encounter 
after training.

When trying to train a model, an important first step is ensuring the quality
of the input data.
Due to the messy and often chaotic nature of data collection in real 
applications, a preprocessing step is necessary to ensure data quality.
Properties of the data such as the mean and standard deviation can tell
us a lot about the nature of the data.
One issue that is not so straightforward to solve is missing data.
Sometimes data is missing completely at random, with no relation 
between the missing pieces of data, while other times some
factor has an influence on the missing attributes.

To address missing data several \emph{imputation} methods have been 
developed, where missing data attributes are filled in using some other
properties of the data.
Some methods rely on statistical properties such as the mean, while
others use clustering techniques to fill in the missing data.
Still others use an entire neural network to estimate the missing values.

The present assignment investigates different data imputation methods
given different datasets and different amounts of missing data in each
to determine the effectiveness of these methods.
First, we introduce the methods used, followed a discussion on the 
experiments.
Finally, we discuss the results of the experiments and provide our
conclusions.

\section{Methods}
\subsection{Mean}
Perhaps the simplest imputation method is using the mean.
In this method, the mean along the columns of the non-missing data
is used to fill in all the missing values of that column.


\section{Experimental Analysis}
\section{Conclusions}

\printbibliography
\end{document}
