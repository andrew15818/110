\documentclass[conference]{IEEEtran}
%\usepackage{cite}
\usepackage[backend=biber,citestyle=ieee]{biblatex}
\usepackage[T1]{fontenc}
\usepackage[utf8]{inputenc}
\usepackage{amsmath, amssymb, amsfonts}
\usepackage{algorithmic}
\usepackage{graphicx}
\usepackage{textcomp}
\usepackage{xcolor}
\addbibresource{report.bib}

% What is this?
\def\BibTeX{{\rm B\kern-.05em{\sc i\kern-.025em b}\kern-.08em
    T\kern-.1667em\lower.7ex\hbox{E}\kern-.125emX}}


\begin{document}
\title{Data Science Final Project Report}
\author{\IEEEauthorblockN{Andr\'es Ponce}
		\IEEEauthorblockA{\textit{Department of Computer Science} \\
		\textit{National Cheng Kung University} \\
		Tainan, Taiwan \\
		andresponce@ismp.csie.ncku.edu.tw}
}
%\author{\IEEEauthorblockN{1\textsuperscript{st} Given Name Surname}
%\IEEEauthorblockA{\textit{dept.\ name of organization (of Aff.)} \\
%\emph{name of organization (of Aff.)}\\
%City, Country \\
%email address or ORCID}
%}

\maketitle
\begin{abstract}
	Many e-commerce platforms need to search for similar or identical products
	given some query image.
	Doing so can increase the platform's ability to recommend interesting 
	products or analyze purchasing trends across product categories.
	For the eBay eProduct Visual Search Challenge, participants take a set
	of query images and search a large index set for matching products.
	We first train a model to recognize the hierarchical structure of the different
	image categories.
	Then, we use our model's output to produce hashes of the index images
	and query images, and locate identical products by comparing these hashes.
	This paper describes our method, experiments, and results used in this
	competition.
\end{abstract}

\section{Introduction}
E-commerce platforms continue to grow and play a large role in consumer's
shopping behavior.
Especially with the pandemic, more people relied on such platforms for 
many of their purchases~\cite{jilkova2021digital}.

E-commerce platforms that allow users to sell their own products especially
require finding images of identical products.
When a user searches for a product, he or she expects the results to contain
images of the same product.
On sites like eBay, identifying identical products can be very useful when 
aggregating sales of different listings of the same product.
Not only do e-commerce platforms rely on such visual search, but also visual
search engines such as Google Images, where the user can use an image as a query
instead of a search term.

The eBay eProduct Visual Recognition Challenge~\cite{jiangbo2021ebay} consists of
finding images of the same product from a large index set of images.
This challenge is one of fine-grained visual classification, since we are trying to 
find images of \emph{the same} product. 
Similar products can differ by very small details, increasing the difficulty of the task.
Likewise, identical products might differ based on lighting conditions or other yet we 
our model should still identify identical products despite these factors.
 pwe are trying to 
 find images of \emph{the same} product. 
 Similar products can differ by very small details, increasing the difficulty of the task.
 Likewise, identical products might differ based on lighting conditions or other
 factors, yet our model should still identify identical products despite these factors.

\printbibliography
\end{document}
