\documentclass[11pt]{scrartcl}
\input{template/structure.tex}
\usepackage{xeCJK}
\usepackage{amsmath}
\usepackage{breqn}
\setCJKmainfont{Noto Serif CJK TC}
\newcommand{\vect}{\boldsymbol}
\title{ 
	\normalfont\normalsize
	\textsc{National Cheng Kung University}\\
	\vspace{25pt}
	\rule{\linewidth}{0.5pt}\\
	\vspace{20pt}
	{\huge Optimization Design Homework 1}\\
	\vspace{12pt}
	\rule{\linewidth}{2pt}\\
	\vspace{12pt}
}
\author{\Large Andr\'es Ponce \and 彭思安 \and P76107116}
\date{\normalsize\today}

\begin{document}
\maketitle
\section{Find the stationary points for the following functions. 
Also identify(for each stationary point), the local maximum, minimum,
or neither (by using second order derivative or Hessian matrix.)}

\subsection{$f(x) = x^{3}\exp(-x^2)$, for $-2<x<2$.}
The stationary points are those where $f'(x) = 0$, so we first find
the first derivative.
By using the product rule,

\begin{equation}
\label{eq:1aFirstDer}
f'(x) = 3x^{2}\exp(-x^{2}) -2x^{4}\exp(-x^{2}) = -x^2 \exp(-x^{2})(2x^4 - 3x^2)
\end{equation}

which is zero at $x=0$, and by solving $2x^4 - 3x^2$ we get the other roots $x=\pm\frac{\sqrt{3}}{\sqrt{2}}$.
Theorem 2.2 gives the sufficient condition for a minimum or maximum point for
single variables.
We must find a point where $f^{m}(x^{*}) \neq 0$, so we find $f''(x)$.
We again use the product rule on Equation~\ref{eq:1aFirstDer} and factor 
the result to obtain

\begin{equation}
	\label{eq:1aSecondDer}
	f''(x) = -x\exp(-x^{2})(4x^4 + 2x^2 -6)
\end{equation}
where $f''(\frac{\sqrt{3}}{\sqrt{2}}) < 0$ and $f''(-\frac{\sqrt{3}}{\sqrt{2}}) > 0$ However, $f''(0) = 0$,
so we find the third derivative by using the same procedure
\begin{equation}
\label{eq:1aThirdDer}
f'''(x) = -x\exp(-x^{2})(20x^3 + 6x + 8x^4 + 4x^2 -12) - 6\exp(-x^2)
\end{equation}
where $f'''(0)=6$, however since $n=3$, $0$ does not correspond to either
a maximum nor minimum point.
In the end, $-\frac{\sqrt{3}}{\sqrt{2}}$ is a minimum, $0$ is neither, and 
$\frac{\sqrt{3}}{\sqrt{2}}$ is a maximum.

\subsection{$f(x, y) = -x^2 -3y^2 + 12xy$}
For a multivariable equation, solving for $x$ and $y$ in the partial derivatives will give us
the stationary points.
\begin{equation}
	\label{eq:1bxFirstDer}
	\frac{\partial f}{\partial x} = -2x + 12y
\end{equation}

\begin{equation}
	\label{eq:1byFirstDer}
	\frac{\partial f}{\partial y} = -6y + 12x
\end{equation}
We first solve for $y$ in Equation~\ref{eq:1byFirstDer} and we get $y=2x$.
Substituing for $y$ in Equation~\ref{eq:1bxFirstDer}, we get 
$$-2x + 24x = 22x$$
$$22x = 0$$
which means $x=0$.
Substituting again $x=0$ in Equation~\ref{eq:1byFirstDer} we get $-6y=0$ which means
$y=0$.
The stationary point for $f$ is $(0, 0)$.

To determine the nature of the stationary points, we have to find the Hessian matrix,
which involves finding the second order partial derivatives.
\begin{equation}
	\frac{\partial^{2} f}{\partial^{2} x}=-2\quad \frac{\partial^{2}f}{\partial^{2}y} = -6
\end{equation}

\begin{equation}
	\frac{\partial^{2} f}{\partial y\partial x}=12\quad \frac{\partial^{2}f}{\partial x\partial y} = 12
\end{equation}
and build the Hessian matrix
$$
\boldsymbol{H} = 
\begin{bmatrix}
	-2 & 12\\
	12 & -6
\end{bmatrix}
$$
We can check for positive or negative definiteness by looking at the sign of $|\boldsymbol{H}_1|$ and $|\boldsymbol{H}_{2}|$.
Here $|\boldsymbol{H}_1| = -2$ and $|\boldsymbol{H}_2| = (-6)(-2) -(12)(12) = -136$.
Since both determinants are negative, we conclude $(0, 0)$ is neither a maximum nor minimum of $f$.

\section{A quadratic function of $n$ variables has the following standard form
		$$f(\vect{x}) = \vect{x}^{T}A\vect{x}/2 + \vect{b}^{T}\vect{x} +c$$, 
		where $\vect{x}$ is the vector containing the $n$ variables.
		Vector $\vect{b}$($n\times1$) and symmetric matrix $\vect{A}$($n\times n$) contain
		constant coefficients.
		For the following two quadratic functions $f_a (\vect{x})=x_{1}^{2}+2x_{1}x_{2}+3x_{2}^{2}$
		and $f_b (\vect{x})=-x_{1}^{2}-x_{2}^{2}-x_{3}^{2}+2x_{1}x_{2}+6x_{1}x_{3}+4x_{1}-5x_{3}+7$, please
}
\subsection{Rewrite these functions in the standard quadratic function form.}
\end{document}
