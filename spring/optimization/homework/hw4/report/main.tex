\documentclass[11pt]{scrartcl}
\input{../template/structure.tex}
\usepackage{xeCJK}
\usepackage{multicol}
%\usepackage{multiline}
\usepackage{amsmath}
\usepackage{breqn}
\usepackage[backend=biber, style=ieee]{biblatex}
\addbibresource{src.bib}

\setCJKmainfont{Noto Serif CJK TC}
\newcommand{\vect}{\boldsymbol}
\title{ 
	\normalfont\normalsize
	\textsc{National Cheng Kung University}\\
	\vspace{25pt}
	\rule{\linewidth}{1pt}\\
	\vspace{20pt}
	{\huge Optimization Design Homework 4}\\
	\vspace{12pt}
	\rule{\linewidth}{2pt}\\
	\vspace{12pt}
}
\author{\Large Andr\'es Ponce \and 彭思安 \and P76107116}
\date{\normalsize\today}

\begin{document}
\maketitle
\section{A cylindrical tank has height $h$ and the radius
of the top and bottom$=r$ (both in meters). What $h$ and $r$
values will minimize the total surface area (including top, bottom, 
and the side) of this cylindrical tank. We are also told the volume
of this tank is to be fixed (i.e. $\pi r^2h=20m^3$). Please
reformulate the objective function to include a penalty term, and solve
the problem numerically using any method we have learned.} 

The surface area of a cylinder is given by 
\begin{equation}
	\label{eq:sarea}
	f(h, r) = 2\pi r(h + r)
\end{equation}
and we are asked to find the values of $h$ and $r$ that minimize $f$ subject to
\[	g(h, r) = \pi r^2h  = 20 \]
which we rewrite as 
\begin{equation}
	\label{eq:constraint}
	g(h, r) = \pi r^2h  -20 = 0
\end{equation}

In order to create an objective function with penalty terms, we combine the 
constraint function and the objective to create 
\begin{equation}
		\label{eq:penalty}
		\varphi(h, r, r_k) = f(h, r) + r_kg(h, r)^2 = 2\pi r(h+r) + r_k(\pi r^2h-20)^2
\end{equation}
In our implementation we use the Nelder-Mead algorithm to minimize this new objective function.
We define a simplex of $n+1$ points and reflect, contract and expand it so that after some iterations
it contains the minimum point inside of it.
When the simplex is small enough, our algorithm terminates since the simplex contains the minimum point.

The first step of the Nelder-Mead algorithm is to find the points that yield the highest, lowest,
and second lowest function value.
Then we find the average point $x_a$ and determine if our point should be reflected, expanded, 
or contracted.
To see if an operation results in a better point for our simplex, we compare the resulting point
with the min and second to last points.

For our algorithm we use $r_k = 1$, and we find the minimum point is roughly $(2.907, 1.454)$.
\section{Solve the following muscle force distribution problem. The $F_i$ values are to be solved.
Values of $A_i, M_i,$ and $d_i$ are given below.}
\subsection{Minimize 
\[Z = \sum_{i=1}^{9}(\frac{F_i}{A_i})^2, n=2\]
subject to 
\[f_1 = d_1F_1 - d_2F_2-d_{3a}F_3-M_1=0\]
\[f_2 = -d_{3k}F_3 + d_4F_4+d_{5k}F_5-d_6F_6-d_{7k}F_7-M_2 = 0\]
\[f_3 = d_{5h}F_5-d_{7h}F_7+d_8F_8-d_9F_9-M_3=0\]
\[F_i \geq 0, i=1, 2, \dots, 9\]
M1=4; M2=33; M3=31;
The values of d1, d2, d3a, d3k, d4, d5k, d5h, d6, d7k, d7h, d8, d9 are included in the following vector: d=[0.0298  0.044  0.044  0.0138  0.0329  0.0329  0.0279  0.025  0.025  0.0619  0.0317  0.0368];
The values of Ai are also included in the vector A=[11.5  92.5  44.3  98.1  20.1  6.1  45.5  31.0  44.3];
}
Following~\cite{RAIKOVA20011243},  we use Lagrange multipliers to minimize this objective.
Our objective follows the form
\begin{equation}
	\label{eq:lagrange}
	L = \sum_{i=1}^{9}(\frac{F_i}{A_i})^2 + \lambda_1f_1 + \lambda_2f_2 + \lambda_2f_3
\end{equation}
We first have to calculate the partial derivatives of $L$ with respect to each muscle force $\frac{\partial L}{\partial F_i}$.
\begin{multicols}{2}
	\begin{equation}
		\label{eq:F1}
		\frac{\partial L}{\partial F_1} = \frac{2F_1}{11.5^2} + 0.0208\lambda_1
	\end{equation}
	\begin{equation}
		\label{eq:F2}
		\frac{\partial L}{\partial F_2} = \frac{2F_2}{22.5^2} - 0.044\lambda_1
	\end{equation}
\end{multicols}
\begin{multicols}{2}
	\begin{equation}
		\label{eq:F3}
		\frac{\partial L}{\partial F_3} = \frac{2F_3}{44.3^2} -0.044\lambda_1-0.0138\lambda_2
	\end{equation}
	\begin{equation}
		\label{eq:F4}
		\frac{\partial L}{\partial F_4} = \frac{2F_4}{49.3^2} - +0.0329\lambda_2 
	\end{equation}
\end{multicols}
\begin{multicols}{2}
	\begin{equation}
		\label{eq:F5}
		\frac{\partial L}{\partial F_5} = \frac{2F_5}{20.1^2} +0.0329\lambda_2+0.0279\lambda_3
	\end{equation}
	\begin{equation}
		\label{eq:F6}
		\frac{\partial L}{\partial F_6} = \frac{2F_6}{6.1^2} - 0.025\lambda_2 
	\end{equation}
\end{multicols}
\begin{multicols}{2}
	\begin{equation}
		\label{eq:F7}
		\frac{\partial L}{\partial F_7} = \frac{2F_7}{45.5^2} -0.025\lambda_2-0.0619\lambda_3
	\end{equation}
	\begin{equation}
		\label{eq:F8}
		\frac{\partial L}{\partial F_8} = \frac{2F_8}{31^2} +0.0317\lambda_3
	\end{equation}
\end{multicols}

\begin{equation}
\label{eq:F9}
\frac{\partial L}{\partial F_9} = \frac{2F_9}{49.3^2} -0.0369\lambda_3
\end{equation}

We can solve for each $F_i$ in Equations~\ref{eq:F1}~-~\ref{eq:F9} and substitute these values 
in the constraints $f_1$ to $f_3$.
When we do so, we obtain
\begin{equation}
		\label{eq:f1fill}
		f_1 = 0.0298(-1.97\lambda_1) - 0.044(11.1375\lambda_1)-0.044(43.17\lambda_1 + 13.52\lambda_2)-4
\end{equation}
%\begin{equation}
		\begin{flalign}
		\label{eq:f2fill}
		f_2 = & -0.0138(43.17\lambda_1 + 13.52\lambda_2) + 0.0329(-0.3806\lambda_2) + 0.0329(-6.65\lambda_2-5.6\lambda_3)- \\
		& 0.025(.46\lambda_2) -0.025(25.88\lambda_2 + 64.07\lambda_3) -33
		\end{flalign}
%\end{equation}
\begin{equation}
	\label{eq:f3fill}
	f_3 = 0.0279(-6.65\lambda_2 -5.6\lambda_3)-0.0619(25.88+64.07\lambda_3)+0.0317(-15.23\lambda_3) - 0.0369(44.8\lambda_3)-31
\end{equation}
We can solve for $\lambda_1$ in Equation~\ref{eq:f1fill}, $\lambda_2$ in Equation~\ref{eq:f2fill} and $\lambda_3$ in Equation~\ref{eq:f3fill}.
\begin{equation}
	\label{eq:lambda1}
	\lambda_1 = -1.97\lambda_2 -1.34
\end{equation}
\begin{equation}
		\label{eq:lambda2}
		-0.6\lambda_1 -1.0805\lambda_2-1.78\lambda_3-33=0
\end{equation}
\begin{equation}
		\label{eq:lambda3}
\lambda_3 = -4.65-.27\lambda_2
\end{equation}

If we substitute Equation~\ref{eq:lambda1} and Equation~\ref{eq:lambda3} into Equation~\ref{eq:lambda2}, we get
$\lambda_1 = -93.5557$,
$\lambda_2 = 46.81$, and 
$\lambda_3 = -17.2887$.

Then we can figure out the values of $F_i$ by substituting into Equations~\ref{eq:F1}-\ref{eq:F9}.
\begin{multicols}{2}
	\begin{equation}
		\label{eq:f1f}
		F_1 = 184.31
	\end{equation}
	\begin{equation}
		\label{eq:f2f}
		F_2 = -1,064.1
	\end{equation}
\end{multicols}
\begin{multicols}{2}
	\begin{equation}
		\label{eq:f3f}
		F_3 = -3,406.06
	\end{equation}
	\begin{equation}
		\label{eq:f4f}
		F_4 = -1,781.21
	\end{equation}
\end{multicols}
\begin{multicols}{2}
	\begin{equation}
		\label{eq:f5f}
		F_5 = 212.69
	\end{equation}
	\begin{equation}
		\label{eq:f6f}
		F_6 = 21.53
	\end{equation}
\end{multicols}
\begin{multicols}{2}
	\begin{equation}
		\label{eq:f7f}
		F_7 = 243.9
	\end{equation}
	\begin{equation}
		\label{eq:f8f}
		F_8 = 111
	\end{equation}
\end{multicols}
\begin{equation}
	\label{eq:f9f}
	F_9 = -775.2
\end{equation}

The results and forms of many of the equations are different than the forms in~\cite{RAIKOVA20011243},
primarily because the form of the Lagrange function is different.
We use the form
\[L(\boldsymbol{F}, \boldsymbol{d}, \boldsymbol{M}, \boldsymbol{A}, \lambda) = Z + \sum_{i}\lambda_ig_i\]
whereas the paper uses
\[L(\boldsymbol{F}, \boldsymbol{d}, \boldsymbol{M}, \boldsymbol{A}, \lambda) = Z - \sum_{i}\lambda_ig_i\]

This means the signs when solving for $\lambda_i$ will be different.
Because some of the forces are negative, violating one of the constraints, $F_2, F_3, F_4, F_5,F_9=0$, meaning they are ``silent''.
\printbibliography
\end{document}

